Mucho de este contenido está basado en
\url{http://en.cship.org/wiki/Special:Allpages}

\section{Administrador de
contraseñas}\label{administrador-de-contraseuxf1as}

Un administrador de contraseñas es software que ayuda al usuario para
organizar sus contraseñas y códigos PIN. El software generalmente tiene
una base de datos local o un archivo que mantiene los datos de las
contraseñas cifrados para conectarse en forma segura con otras
computadoras, redes, sitios web y archivos de aplicaciones. KeePass
http://keepass.info/ es un ejemplo.

\subsection{Agregador}\label{agregador}

Un agregador es un servicio que ofrece información recolectada de un
sitio y lo pone disponible en diferentes direcciones. Se lo conoce
también como agregador RSS, agregador de feeds, lector de feeds, o
lector de noticias (No se debe confundir con el lector de noticias de
Usenet)

\subsection{Análisis de amenazas}\label{anuxe1lisis-de-amenazas}

Un análisis de amenazas a la seguridad es un estudio formal, adecuado al
detalle, de todas las maneras conocidas de ataques a la seguridad de los
servidores o los protocolos, o de los métodos usados para un propósito
particular tales como evasión. Las amenazas pueden ser de carácter
técnico, como romper el código o explotar los errores de software, o
sociales, como el robo de contraseñas o sobornar a alguien que tiene un
conocimiento especial. Pocas compañías o individuos tienen el
conocimiento y la habilidad para hacer un análisis global, pero todos
los implicados tienen que hacer alguna estimación de los temas.

\subsection{Análisis de tráfico}\label{anuxe1lisis-de-truxe1fico}

El análisis de tráfico consiste en el análisis estadístico de las
comunicaciones cifradas. En algunas circunstancias puede revelar
información acerca de la gente comunicada y la información que están
compartiendo.

\subsection{Ancho de banda}\label{ancho-de-banda}

El ancho de banda de una conexión es la máxima velocidad de
transferencia de datos, limitada por su capacidad y las características
de las computadoras en ambos extremos de la conexión.

\subsection{Anonimato}\label{anonimato}

(No se debe confundir con privacidad, seudoanonimato, seguridad o
confidencialidad)

El anonimato en Internet es la capacidad de utilizar los servicios sin
dejar pistas sobre su identidad o sin ser espiado. El nivel de
protección depende de las técnicas de anonimato utilizados y el grado de
seguimiento. Los más fuertes técnicas en uso para proteger el anonimato
implican la creación de una cadena de comunicación a través de un
proceso aleatorio para seleccionar algunos de los enlaces, en la que
cada eslabón tiene acceso a la información parcial sobre el proceso. El
primero conoce la dirección del usuario de Internet (IP), pero no el
contenido, el destino o finalidad de la comunicación, ya que el
contenido del mensaje e información de destino están cifrados. El último
conoce la identidad del sitio que se está en contacto, pero no la fuente
de la sesión. Algunos pasos intermedios entre los enlaces impiden que el
primero y el último compartan su conocimiento parcial con el fin de
conectar el usuario y el sitio de destino.

\subsection{Archivo de registro}\label{archivo-de-registro}

Un archivo de registro es un archivo que guarda una secuencia de
mensajes enviados por algún proceso de software, el cual puede ser una
aplicación o un componente del sistema operativo. Por ejemplo, los
servidores web o los proxies pueden mantener registros que contienen
información acerca de cuáles direcciones IP usan estos servicios y
cuándo acceden y a que páginas.

\subsection{ASP (proveedor de servicios de
aplicaciones)}\label{asp-proveedor-de-servicios-de-aplicaciones}

Un ASP es una organización que ofrece software sobre Internet,
permitiendo actualizarlo y mantenerlo en forma centralizada.

\subsection{Ataque por fuerza bruta}\label{ataque-por-fuerza-bruta}

Un ataque por fuerza bruta consiste en tratar de averiguar una
contraseña probando todos las variantes posibles. Es uno de los ataques
de hacking más básicos.

\section{Backbone}\label{backbone}

Un backbone (a veces llamado red troncal) es uno de los enlaces de
comunicaciones de gran ancho de banda que une redes en diferentes países
y organizaciones alrededor del mundo en Internet.

\subsection{Badware}\label{badware}

Consulte \emph{malware}.

\subsection{Bash (Bourne-again shell)}\label{bash-bourne-again-shell}

El shell bash es una interfaz de línea de comandos para sistemas
operativos GNU/Linux o Unix, basado en el shell Bourn.

\subsection{BitTorrent}\label{bittorrent}

BitTorrent es un protocolo para compartir archivos entre pares,
inventado por Bram Cohen en 2001. Permite a los individuos distribuir de
forma barata y eficaz archivos de gran tamaño, como imágenes de CD,
video o archivos de música.

\subsection{Bluebar}\label{bluebar}

La barra azul de URL (llamada en la jerga Bluebar Psiphon) es la forma
en la parte superior de la ventana del navegador del nodo Psiphon que le
permite acceder al sitio bloqueado escribiendo su URL en el interior.

Vea también \emph{nodo Psiphon}.

\subsection{Bloqueo}\label{bloqueo}

El bloqueo impide el acceso a un recurso de Internet, basado en un gran
número de métodos.

\section{Caché}\label{cachuxe9}

La caché es una parte de un sistema de procesamiento de información
usada para almacenar datos usados en forma reciente o muy frecuente con
el fin de acelerar el acceso repetido a ellos. Una caché web mantiene
copias de los archivos de la página web.

\subsection{Censorware}\label{censorware}

Censorware es software usado para filtrar o bloquear el acceso a
Internet. Este término se usa a menudo para referirse al software
instalado en la máquina cliente (la computadora usada para acceder a
Internet). La mayoría del censorware se utiliza con propósitos de
control parental. Algunas veces el término censorware se utiliza también
para referirse al software usado con los mismos propósitos pero
instalado en un servidor de red o un router.

\subsection{Censura}\label{censura}

Censurar es evitar la publicación o recuperación de información, o tomar
medidas, legales o de otro tipo, contra los editores y lectores.

\subsection{CGI (Interfaz de gateway
común)}\label{cgi-interfaz-de-gateway-comuxfan}

CGI es un estándar de uso común que permite a los programas de un
servidor web ejecutarse como aplicaciones. Algunas páginas web basadas
en proxy usan CGI, por lo que se denominan ``proxies CGI''. (Una de las
más populares aplicaciones escrita por James Marshall usa el lenguaje de
programación Perl y se denomina CGIProxy.)

\subsection{Cifrado}\label{cifrado}

Se denomina así a todo método usado para recodificar y mezclar datos o
transformarlos matemáticamente para que sea ilegible a las terceras
partes y por lo tanto, no puedan descifrar el secreto que oculta. Es
posible cifrar datos en su disco rígido local usando software como
TrueCrypt (http://www.truecrypt.org) o cifrar el tráfico de Internet con
TLS/SSL o SSH.

vea también \emph{descifrado}.

\subsection{Cifrado completo de disco}\label{cifrado-completo-de-disco}

vea \emph{cifrado de disco}.

\subsection{Cifrado de disco}\label{cifrado-de-disco}

El cifrado de disco es una tecnología que protege la información al
convertirla en ilegible para que no pueda ser descifrada fácilmente por
personas no autorizadas. Usa software o hardware para cifrar cada bit de
datos que está en un disco o en un volumen de disco. El cifrado de disco
previene el acceso no autorizado a los datos almacenados.

\subsection{Clave pública}\label{clave-puxfablica}

vea \emph{criptografía de clave pública/cifrado de clave pública}.

\subsection{Código (de cifrado)}\label{cuxf3digo-de-cifrado}

En criptografía, un código es un algoritmo para realizar cifrado o
descifrado de mensajes.

\subsection{Confidencialidad directa perfecta
(PFS)}\label{confidencialidad-directa-perfecta-pfs}

En un protocolo de acuerdo de claves autenticadas que utiliza la
criptografía de clave pública, la confidencialidad directa perfecta (PFS
o) es la propiedad que asegura que una clave de sesión derivada de un
conjunto de claves públicas y privadas de largo plazo no se verá
comprometida si una de las claves privadas (a largo plazo) se ve
comprometida en el futuro.

\subsection{Cookie}\label{cookie}

Un cookie es una cadena de texto enviado por un servidor web al
navegador del usuario para almacenarla en su computadora, conteniendo
información necesaria para mantener la continuidad en las sesiones a
través de múltiples páginas web, o a través de múltiple sesiones.
Algunos sitios web no se pueden usar sin aceptar ni almacenar una
cookie. Algunas personas consideran esto como una invasión a la
privacidad y/o un riesgo de seguridad.

\subsection{Criptografía}\label{criptografuxeda}

La criptografía es la práctica y el estudio de las técnicas para
establecer comunicaciones seguras en presencia de terceras partes (los
llamados adversarios). En forma más general, consiste en la construcción
y análisis de protocolos para vencer a nuestros adversarios en varios
aspectos relacionados con la seguridad de la información tales como
confidencialidad, integridad, autentificación y no repudio de datos. La
criptografía moderna abarca un amplio rango de disciplinas tales como
matemáticas, ciencias de la computación e ingeniería eléctrica. Sus
aplicaciones incluyen tarjetas ATM, contraseñas de computadoras y
comercio electrónico.

\subsection{Criptografía de clave pública/cifrado de clave
pública}\label{criptografuxeda-de-clave-puxfablicacifrado-de-clave-puxfablica}

La criptografía de clave pública se refiere al sistema de cifrado que
requiere dos claves separadas, una de las cuales es secreta y la otra
pública. Aunque son diferentes, ambas claves están relacionadas
matemáticamente. Una clave cifra el texto plano y la otra lo descifra.
Ninguna clave puede realizar ambas funciones. Una de ellas puede ser
publicad, mientras que la otra debe mantenerse en privado.

La criptografía de clave pública usa algoritmos de clave asimétricos
(tales como RSA), y se suele llamar por el término mas general
\emph{criptografía de clave asimétrica}.

\subsection{Clave privada}\label{clave-privada}

vea \emph{criptografía de clave pública/cifrado de clave pública}.

\section{Chat}\label{chat}

El chat, también llamado mensajería instantánea, es un método habitual
de comunicación entre dos o más personas en la cual cada línea tipeada
por un participante en una sesión es vista por los otros. Existen
numerosos protocolos, incluyendo aquellos creados por empresas
específicas (AOL, Yahoo!, Microsoft, Google, y otros) y los definidos
públicamente. Algunos clientes soportan un único protocolo, pero la
mayoría utiliza una variedad de los protocolos más populares.

\section{DARPA}\label{darpa}

DARPA (Defense Advanced Projects Research Agency, Agencia de
investigación de proyectos avanzados de defensa) es el sucesor de ARPA,
que fundó Internet y su predecesor, ARPAnet.

\subsection{Descifrado}\label{descifrado}

Descifrar es recuperar el texto plano u otros mensajes a partir de un
mensaje cifrado mediante el uso de una clave.

Consulte también \emph{cifrado}.

\subsection{Dirección IP (dirección del protocolo de
Internet)}\label{direcciuxf3n-ip-direcciuxf3n-del-protocolo-de-internet}

Una dirección IP es un número que identifica una computadora en
particular en Internet. En la versión 4 (IPv4) consiste en cuatro bytes
(32 bits), a menudo representada por cuatro números enteros del rango de
0 a 255 separados por puntos, tal como 74.54.30.85. En IPv6, versión a
la cual actualmente está cambiando la red, una dirección IP es cuatro
veces más larga, y consiste de 128 bits. Puede ser escrita en 8 grupos
de de 4 dígitos hexadecimales separados por dos puntos, por ejemplo
2001:0db8:85a3:0000:0000:8a2e:0370:7334.

\subsection{Dirección IP públicamente
ruteable}\label{direcciuxf3n-ip-puxfablicamente-ruteable}

Las direcciones IP públicamente ruteables (a veces llamadas direcciones
IP públicas) son aquellas que pueden alcanzarse de forma normal en
Internet, a través de una cadena de enrutadores. Algunas direcciones IP
son privadas, como el bloque 192.168.x.x, y muchas no están asignadas.

\subsection{DNS (Sistema de nombres de
dominio)}\label{dns-sistema-de-nombres-de-dominio}

El sistema de nombres de dominio (DNS) convierte los nombres de dominio,
compuestos por combinaciones de letras fáciles de recordar, a las
direcciones IP, que son cadenas de números difíciles de recordar. Cada
computadora en Internet tiene una dirección única (algo parecido a un
código de área + número telefónico)

\subsection{Dominio}\label{dominio}

Un dominio puede ser un dominio de nivel superior (TLD) o un dominio
secundario de Internet.

Vea también \emph{dominio de nivel superior}, \emph{dominio de nivel
superior con código país} y \emph{dominio secundario}.

\subsection{Dominio de alto nivel con código de país
(ccTLD)}\label{dominio-de-alto-nivel-con-cuxf3digo-de-pauxeds-cctld}

Cada país tiene un código de dos letras, y un TLD (dominio de alto
nivel) basado en él, tal como .ca para Canada; este dominio se llama
dominio de alto nivel con código de país. Cada ccTLD tiene un servidor
DNS que lista todos los dominios de segundo nivel dentro del TLD. Los
servidores raíz apuntan a todos los TLD, y almacenan la información
usada frecuentemente en los dominios de nivel inferior.

\subsection{Dominio de nivel superior
(TLD)}\label{dominio-de-nivel-superior-tld}

En el ámbito de los nombres de Internet, el TLD es el último componente
del nombre de dominio. Existen diversos TLD genéricos, los más
importantes son .com, .org, .edu, .net, .gov, .mil, .int, y un código de
país de dos letras (ccTLD) diferente para cada uno de ellos, por
ejemplo, .ca for Canada. La Unión Europea también tiene código propio,
.eu.

\section{E-mail (correo
electrónico)}\label{e-mail-correo-electruxf3nico}

E-mail, abreviatura en inglés de correo electrónico, es un método para
enviar y recibir mensajes por Internet. Se puede usar un servicio de web
mail o enviarlos con el protocolo SMTP y recibirlos con el protocolo
POP3 mediante un cliente de correo electrónico tal como Outlook Express
o Thunderbird. Es raro que un gobierno bloquee el correo electrónico,
sin embargo, la vigilancia es muy común. Si el correo electrónico no
está cifrado, será muy fácil de leer por algún operador de red o un
gobierno.

\subsection{Escuchas ilegales}\label{escuchas-ilegales}

Las escuchas ilegales consisten en interceptar al tráfico de voz o la
lectura o filtrar el tráfico de datos en una línea telefónica o una
conexión de datos digitales, por lo general para detectar o prevenir
actividades ilegales o no deseadas o para controlar o monitorear lo que
la gente está hablando.

\subsection{Esquema}\label{esquema}

En la Web, un esquema es una asignación de un nombre a un protocolo.
Así, el esquema HTTP asigna URLs que comienzan con HTTP: el Protocolo de
Transferencia de Hipertexto. El protocolo determina la interpretación
del resto de la URL, por lo que http://www.example.com/dir/content.html
identifica un sitio web y un archivo específico en un directorio
específico, y mailto: user@somewhere.com es una dirección de correo
electrónico de una persona o grupo específico en un dominio específico.

\subsection{Esteganografía}\label{esteganografuxeda}

Esta palabra, que en griego significa escritura ocultar, se refiere a la
variedad de métodos para enviar mensajes ocultos donde no sólo lo está
el contenido, sino que también es muy probable que algo que lo encubra
también lo esté. Generalmente se oculta una cosa dentro de otra, como
una fotografía o un texto dentro de algo sin relación alguna.
Contrariamente a la criptografía, donde está claro que se está
transmitiendo un mensaje secreto, la estenografía intenta ocultar
también a la comunicación en sí misma.

\subsection{Evasión}\label{evasiuxf3n}

La evasión es publicar o dar acceso a los contenidos, a pesar de los
intentos de censura.

\subsection{Expresión regular}\label{expresiuxf3n-regular}

Una expresión regular (también conocida como regexp o RE) es un patrón
de texto que especifica un conjunto de cadena de textos en una
implementación particular de una expresión regular tal como la utilidad
grep de UNIX. Una cadena de texto ``encuentra'' una expresión regular si
la cadena concuerda con el patrón, como está definido en la sintaxis de
la expresión regular. En cada sintaxis de RE, algunos caracteres tienen
significados especiales, para permitir que un patrón encuentre múltiples
cadenas. Por ejemplo, la expresión regular lo+se encuentra lose, loose,
and looose.

\section{Filtro}\label{filtro}

Es alguna forma de búsqueda de patrones de datos para bloquear o
permitir las comunicaciones.

\subsection{Filtro de bajo ancho de
banda}\label{filtro-de-bajo-ancho-de-banda}

Un filtro de bajo ancho de banda es un servicio web que remueve
elementos extraños tales como publicidades e imágenes de una página web
y además la comprime, haciendo mucho más rápida la descarga.

\subsection{Filtro de palabra clave}\label{filtro-de-palabra-clave}

Un filtro de palabra clave escanea todo el tráfico de Internet que pasa
a través de un servidor para hallar palabras o términos prohibidos para
poder bloquearlas.

\subsection{Firefox}\label{firefox}

Firefox es el navegador web open source más popular, desarrollado por la
Fundación Mozilla.

\subsection{Foro}\label{foro}

En un sitio web, un foro es un lugar para la discusión, donde los
usuarios pueden publicar mensajes y comentar otros previamente
publicados. Se diferencia de una lista de correo o un grupo de Usenet
por la persistencia de las páginas que contienen los hilos de los
mensajes. Los archivos de grupo de noticias y listas de correo, sin
embargo, muestran habitualmente un mensaje por página, con páginas de
navegación que listan solamente las cabeceras de los mensajes en un
hilo.

\subsection{Frame (marco)}\label{frame-marco}

Un frame es una porción de una página web que posee su propia URL. Por
ejemplo, los frames se usan habitualmente para colocar un menú estático
cercano a una ventana con texto deslizante.

\subsection{FTP (Protocolo de transferencia de
archivo)}\label{ftp-protocolo-de-transferencia-de-archivo}

El protocolo FTP se usa para transferir archivos. Mucha gente lo usa
generalmente para descargas; aunque se puede usar también para cargar
páginas web y scripts para algunos servidores web. Usa habitualmente los
puertos 20 y 21, que a veces están bloqueados. Algunos servidores FTP
escuchan en otros puertos, pudiendo evadir el bloqueo basado en puertos.

Un cliente FTP popular libre para Windows y Mac OS es FileZilla. Existen
también algunos clientes FTP basados en la web que pueden usarse con un
navegador normal tal como Firefox.

\subsection{Fuga de DNS}\label{fuga-de-dns}

Una fuga de DNS ocurre cuando un equipo que está configurado para usar
un proxy para su conexión a Internet, sin embargo hace consultas DNS sin
usarlo, lo que expone a los intentos de los usuarios para conectarse con
sitios bloqueados. Algunos navegadores web tienen opciones de
configuración para forzar el uso del proxy.

\section{Gateway}\label{gateway}

Un gateway es un nodo que conecta dos redes en Internet. Un ejemplo
importante son los gateways nacionales a través de los cuales pasa todo
el tráfico, tanto entrante como saliente.

\subsection{GNU Privacy Guard}\label{gnu-privacy-guard}

GNU Privacy Guard (GnuPG o GPG) es una aplicación de software de
criptografía, alternativa de PGP, con licencia libre GPL. Cumple con la
especificación RFC 4880, la especificación estándar actual IETF de
OpenPGP.

vea también \emph{PGP}.

\subsection{GPG}\label{gpg}

vea \emph{GNU Privacy Guard}.

\section{Honeypot}\label{honeypot}

Un honeypot es un sitio web que simula ofrecer un servicio para tentar a
usuarios potenciales para que lo usen, y poder capturar información
sobre ellos o sus actividades.

\subsection{HTTP (Protocolo de transferencia de
hipertexto)}\label{http-protocolo-de-transferencia-de-hipertexto}

HTTP es el protocolo fundamental de la World Wide Web, que proporciona
métodos para solicitar y mostrar páginas Web, consultar y generar
respuestas a las consultas, y el acceso a una amplia gama de servicios.

\subsection{HTTPS (HTTP seguro)}\label{https-http-seguro}

Es un protocolo de comunicación segura mediante cifrado de mensajes
HTTP. Los mensajes entre el cliente y el servidor se cifran en ambas
direcciones, utilizando claves generadas cuando la conexión se solicitó
y se intercambiaron con seguridad. Las direcciones IP de origen y
destino están en las cabeceras de cada paquete, así que HTTPS no puede
ocultar el hecho de la comunicación, sólo el contenido de los datos
transmitidos y recibidos.

\section{IANA}\label{iana}

IANA (Internet Assigned Numbers Authority, autoridad de asignación de
números de internet) es la organización responsable de los trabajos
técnicos en la gestión de la infraestructura de Internet, incluyendo la
asignación de bloques de direcciones IP para dominios de nivel superior
y los registradores de licencias de dominio para los ccTLD y de los
dominios de nivel superior genéricos, la ejecución de los servidores
raíz de Internet, y otros funciones.

\subsection{ICANN}\label{icann}

ICANN (Internet Corporation for Assigned Names and Numbers, corporación
de internet para nombres y números asignados) es una corporación creada
por el Departamento de Comercio de los EEUU para administrar los niveles
más altos de Internet. El trabajo técnico lo lleva a cabo IANA.

\subsection{Mensajería instantánea
(IM)}\label{mensajeruxeda-instantuxe1nea-im}

La mensajería instantánea se refiere a chatear usando protocolos
propietarios, o a chatear en general. Los clientes de mensajería
instantánea más comunes son MSN Messenger, ICQ, AIM o Yahoo! Messenger.

\subsection{Intermediario}\label{intermediario}

vea \emph{man in the middle}.

\subsection{Intercambio de archivos}\label{intercambio-de-archivos}

El intercambio de archivos se refiere a cualquier sistema de computadora
donde mucha gente puede usar la misma información, pero a menudo se
refiere a música, películas u otros materiales disponibles libres de
cargo en Internet.

\subsection{Interfaz común de
gateway}\label{interfaz-comuxfan-de-gateway}

vea \emph{CGI}.

\subsection{Interfaz de línea de
comandos}\label{interfaz-de-luxednea-de-comandos}

Es un método para controlar la ejecución de software usando comandos
ingresados con un teclado, tal como un shell de Unix o una línea de
comandos de Windows.

\subsection{Internet}\label{internet}

Internet es una red de redes interconectadas que usan TCP/IP y otros
protocolos de comunicación.

\subsection{IRC (Internet relay chat)}\label{irc-internet-relay-chat}

IRC es un protocolo de Internet de más de 20 años de antiguedad usado
para conversaciones de texto en tiempo real (chat o mensajería
instantánea). Existen distintas redes IRC, la mayor posee más de 50.000
usuarios.

\subsection{ISP (Proveedor de servicio de
internet)}\label{isp-proveedor-de-servicio-de-internet}

Un ISP (proveedor de servicio de Internet) es una empresa u organización
que suministra acceso a Internet para sus clientes.

\section{JavaScript}\label{javascript}

JavaScript es un lenguaje de scripting, de uso habitual en las páginas
web que suministran funciones interactivas.

\section{KeePass, KeePassX}\label{keepass-keepassx}

KeePass y KeePassX son dos tipos de administradores de contraseñas.

\section{Latencia}\label{latencia}

La latencia es una medida del tiempo de demora experimentado en un
sistema, en este contexto, en una computadora en la red. Se mide como el
tiempo transcurrido entre el comienzo de la transmisión de un paquete y
el comienzo de su recepción, entre un extremo de la red (por ejemplo
usted) y el otro (por ejemplo el servidor web). Una manera muy poderosa
de filtrado web es mantener una muy alta latencia, la que provoca que el
uso de muchas herramientas de evasión sea muy dificultoso.

\subsection{Lista blanca}\label{lista-blanca}

Una lista blanca (whitelist) es una lista de sitios específicamente
autorizados para establecer alguna forma particular de comunicación. Se
puede filtrar el tráfico con una lista blanca (bloqueando todo excepto
los sitios de la lista), una lista negra (permitiendo todo excepto los
sitios de la lista), una combinación de ambas u otras políticas basadas
en reglas y condiciones específicas.

\subsection{Lista negra}\label{lista-negra}

Una lista negra (blacklist) es una lista de cosas prohibidas. Para la
censura en Internet, es una lista de los sitios web o las direcciones IP
de las computadoras prohibidas; se permite acceder a todos los sitios
y/o computadoras excepto a aquellos listados específicamente. Una
alternativa es una lista blanca, o lista de cosas permitidas. Una lista
blanca bloquea el acceso a todos los sitios excepto a aquellos
específicamente listados. No es muy común. Es posible combinar ambos
tipos de listas usando cadenas de búsqueda u otras técnicas
condicionales en URL que no coincidan con ninguna de las listas.

\section{Malware}\label{malware}

Malware es un término general para referirse a software malicioso,
incluyendo a los virus, que pueden estar instalados o pueden ser
ejecutados sin su conocimiento. El malware toma el control de su
computadora para fines específicos como, por ejemplo, enviar spam.
(También se conoce al malware como badware.)

\subsection{Man in the middle}\label{man-in-the-middle}

Un man in the middle (\emph{hombre en el medio}) es una persona o
computadora que captura tráfico en un canal de comunicación,
principalmente para realizar cambios selectivos o bloquear el contenido
de una manera que socave la seguridad criptográfica. En general, el
ataque man-in-thr-middle implica pasar por un sitio Web, servicio o
individuo con el fin de registrar o alterar las comunicaciones. Los
gobiernos pueden ejecutar man-in-thr-middle en los gateways de entrada a
un país por donde pasa todo el tráfico.

\subsection{Marcador}\label{marcador}

Un marcador es una referencia a una posición dentro del software que
apunta a un recurso externo. En un navegador, es una referencia a una
página web -- al elegir un marcador usted puede cargar rápidamente el
sitio web sin necesidad de tipear la URL completa.

\subsection{Monitoreo}\label{monitoreo}

El monitoreo consiste en el control continuo de un flujo de datos en
busca de actividad no deseada.

\subsection{Motor de difusión de
archivos}\label{motor-de-difusiuxf3n-de-archivos}

Un motor de difusión de archivos es un sitio web editor que puede ser
usado para eludir la censura. Un usuario sólo tiene que cargar el
archivo a publicar una vez y el motor lo propaga a un conjunto de
servicios de almacenamiento compartido (como Rapidshare o Megaupload).

\section{NAT (Traducción de dirección de
red)}\label{nat-traducciuxf3n-de-direcciuxf3n-de-red}

NAT es una función de un router para ocultar un espacio de direcciones
de reasignación. Todo el tráfico que sale del router, utiliza su
dirección IP, y el router sabe cómo enrutar el tráfico entrante a quien
se lo solicite. NAT es frecuentemente aplicado por los cortafuegos.
Puesto que las conexiones entrantes son normalmente prohibidas por NAT,
se hace difícil ofrecer un servicio al público en general, como un sitio
Web o un proxy público. En una red donde NAT está en uso, ofrecer este
servicio requiere algún tipo de configuración de cortafuegos o método
NAT transversal.

\subsection{Nodo}\label{nodo}

Un nodo es un dispositivo activo en una red. Un router es un ejemplo de
un nodo. En las redes Psiphon y Tor, un servidor se conoce también como
nodo.

\subsection{Nodo abierto}\label{nodo-abierto}

Un nodo abierto es un nodo específico Psiphon que puede ser usado sin
loguearse. Éste carga automáticamente una página de inicio propia, y se
presenta a sí mismo en un lenguaje propio, pero puede ser usado por
cualquier navegador web.

vea también \emph{nodo Psiphon}.

\subsection{Nodo de enlace o
intermedio}\label{nodo-de-enlace-o-intermedio}

Un nodo intermedio es un nodo Tor que no es un nodo de salida. La
ejecución de un nodo intermedio puede ser más segura que la ejecución de
un nodo de salida, porque un nodo intermedio no se mostrará en los
archivos de registro de terceros. (Un nodo intermediario a veces se
llama un nodo sin salida.)

\subsection{Nodo de salida}\label{nodo-de-salida}

Un nodo de salida es un nodo Tor que reenvía datos fuera de la red Tor.

\subsection{Nodo privado}\label{nodo-privado}

Un nodo privado es un nodo Psiphon que trabaja con autentificación, lo
que significa que usted debe registrarse antes de poder usarlo. Hecho
esto, podrá enviar invitaciones a su amigos para que usen este nodo
específico. Vea también \emph{nodo Psiphon}.

\subsection{Nodo Psiphon}\label{nodo-psiphon}

Un nodo Psiphon es un proxy web seguro diseñado para evadir la censura
en Internet. Fue desarrollado por Psiphon inc. Psiphon puede ser de
código libre o privativo.

\subsection{Nodo sin salida}\label{nodo-sin-salida}

vea \emph{nodo de enlace o intermedio}.

\section{Ofuscación}\label{ofuscaciuxf3n}

La ofuscación consiste en ocultar texto utilizando técnicas de
transformación fáciles de entender y de revertir que resistan la
inspección casual, pero no al criptoanálisis, o hacer cambios menores en
las cadenas de texto para prevenir comparaciones simples. Los proxies
Web suelen utilizar la ofuscación para ocultar ciertos nombres y
direcciones de los filtros de texto simples que pueden ser engañados.
Por ejemplo, cualquier nombre de dominio puede contener opcionalmente un
punto final, como en ``somewhere.com.'', Pero algunos filtros pueden
buscar sólo ``somewhere.com'' (sin el punto final).

\subsection{Operador de red}\label{operador-de-red}

Un operador de red es una persona u organización que mantiene o controla
una red y se encuentra en posición de monitorear, bloquear o alterar la
comunicación que pasa a través de su red.

\subsection{OTR (mensajes sin
registro)}\label{otr-mensajes-sin-registro}

Un mensaje sin registro, comúnmente denominado OTR, es un protocolo
criptográfico que suministra un cifrado fuerte para conversaciones de
mensajería instantánea.

\section{Paquete}\label{paquete}

Un paquete es una estructura de datos definida por un protocolo de
comunicación que contiene información específica en formas
predeterminadas junto con datos arbitrarios para ser comunicados de un
punto a otro. Los mensajes se dividen en partes que se almacenan en
paquetes para ser transmitidos y luego se ensamblan en el otro extremo
del enlace.

\subsection{Pastebin}\label{pastebin}

Es un servicio web donde cualquier tipo de texto puede ser cargado y
leído sin tener que registrarse. Todos los textos son visibles
públicamente.

\subsection{P2P}\label{p2p}

Una red de pares (o P2P, \emph{peer to peer}) es una red de computadoras
entre iguales. A diferencia de las redes cliente-servidor no hay un
servidor central por lo que el tráfico se distribuye sólo entre
clientes. Esta tecnología se aplica sobre todo en los programas de
intercambio como BitTorrent, eMule y Gnutella. Pero también la
tecnología del muy antiguo Usenet o del programa de VoIP Skype VoIP se
pueden clasificar como sistemas P2P.

vea también \emph{compartir archivos}.

\subsection{\texorpdfstring{PGP (Pretty Good Privacidad,
\emph{privacidad bastante
buena})}{PGP (Pretty Good Privacidad, privacidad bastante buena)}}\label{pgp-pretty-good-privacidad-privacidad-bastante-buena}

PGP es un programa de computadora para cifrado que suministra privacidad
criptográfica y autentificación para comunicación de datos. Se utiliza a
menudo para firmar, cifrar y descifrar textos, correos electrónicos,
archivos, directorios y particiones de discos para incrementar la
seguridad de las comunicaciones por correo electrónico.

PGP y otros productos similares siguen el estándar OpenPGP (RFC 4880)
para cifrado y descifrado de datos.

\subsection{PHP}\label{php}

PHP es un lenguaje de scripting diseñado para crear sitios web dinámicos
y aplicaciones web. Se instala en un servidor. Por ejemplo, el popular
proxy web PHProxy usa esta tecnología.

\subsection{POP3}\label{pop3}

El protocolo POP3 (Post Office Protocol version 3) es usado para recibir
correos electrónicos de un servidor, por defecto en el puerto 110 con un
programa de correo electrónico tal como Outlook Express o Thunderbird.

\subsection{Privacidad}\label{privacidad}

La protección de la intimidad consiste en impedir la divulgación de
información privada personal sin el consentimiento de la persona
interesada. En este contexto, significa impedir que algunos observadores
se enteren de que una persona haya solicitado o recibido información que
ha sido bloqueada o es ilegal en el país donde se encuentre la persona
en cuestión.

\subsection{Protocolo}\label{protocolo}

Una definición formal de un método de comunicación, y la forma en que
los datos deben ser transmitidos. Además, se refiere al propósito de tal
método de comunicación. Por ejemplo, el protocolo para la transmisión de
paquetes de datos en Internet (IP), o el protocolo de transferencia de
hipertexto para las interacciones en la World Wide Web (HTTP).

\subsection{Proxy Web}\label{proxy-web}

Un proxy web es un script que se ejecuta en un servidor que actúa como
un proxy/gateway. Los usuarios pueden acceder al proxy web con su
navegador habitual (por ejemplo Firefox) e ingresar cualquier URL en el
formulario localizado en el sitio web. luego el programa del servidor
recibirá el contenido y lo mostrará al usuario. De esta forma el ISP
sólo verá una conexión al servidor con el proxy web ya que no se ha
establecido una conexión directa.

\subsection{Puente}\label{puente}

Vea \emph{Puente Tor}

\subsection{Puente Tor}\label{puente-tor}

Un puente es un nodo intermedio que no está listado en el directorio
público principal de Tor, por lo que es especialmente útil en países
donde las comunicaciones están bloqueadas. A diferencia del caso de los
nodos de salida, las direcciones IP de los nodos puente nunca aparecen
en los archivos de registro del servidor y nunca pasan a través de los
nodos de control de manera que pueden ser conectados con la evasión.

\subsection{Puerto}\label{puerto}

Un puerto de hardware en una computadora es un conector físico para un
propósito específico que usa un protocolo de hardware propio. Algunos
ejemplos son el puerto de la pantalla VGA o un conector USB.

Los puertos de software también conectan computadoras y otros
dispositivos en las redes usando distintos protocolos, pero existen en
el software solamente como números. Los puertos son algo así como los
números puestos sobre las puertas que dan acceso a distintas
habitaciones, cada uno con un servicio especial en un servidor o en una
PC. Están identificados por números enteros entre 0 y 65535.

\section{Remailer}\label{remailer}

Un remailer anónimo es un servicio que le permite a los usuarios enviar
correos electrónicos anónimamente. El remailer recibe los mensajes y lo
reenvía a su destinatario después de remover la información que podría
identificar al remitente original. Algunos servicios también proveen una
dirección anónima que puede ser usada para recibir las respuestas sin
descubrir su identidad. Algunos servicios de remailer conocidos incluyen
a Cypherpunk, Mixmaster y Nym.

\subsection{Router}\label{router}

Un router es una computadora que determina la ruta para reenviar
paquetes. Utiliza la información de la dirección en la cabecera del
paquete y la información de la caché en el servidor para hallar los
números de dirección con conexiones de hardware.

\subsection{RSS (agregador de
noticias)}\label{rss-agregador-de-noticias}

RSS es un método y un protocolo que le permite a los usuarios de
Internet suscribirse al contenido de una página web, y recibir
actualizaciones tan pronto como sean publicadas.

\section{Salto (Hop)}\label{salto-hop}

Es un enlace en una cadena de paquetes transferidos desde una
computadora a otra, o alguna computadora a lo largo de la ruta. El
número de saltos entre computadoras puede brindar una estimación de la
demora (latencia) en las comunicaciones entre ellas. Cada salto
individual es también una entidad que posee la capacidad de escuchar,
bloquear o alterar las comunicaciones.

\subsection{Screenlogger}\label{screenlogger}

Un screenlogger es un software capaz de registrar todo lo que su
computadora muestra en la pantalla. Su principal característica es
capturar la pantalla y el login en archivos para consultarlos en otro
momento. Los screenloggers pueden utilizarse como una poderosa
herramienta de monitoreo. Debe ser precavido con todas las pantallas de
login que se ejecuten en la computadora que esté usando, en todo
momento.

\subsection{Script}\label{script}

Un script es un programa, generalmente escrito en un lenguaje
interpretado, no compilado (tal como JavaScript o Java), o en un
lenguaje interpretado de comandos tal como bash. Muchas páginas web
incluyen scripts para administrar la interacción con ella, y entonces el
servidor no necesita reenviar cada página ante un nuevo cambio.

\subsection{Script embebido}\label{script-embebido}

Un script embebido es una pieza de código de software.

\subsection{Servidor de nombre raíz}\label{servidor-de-nombre-rauxedz}

Un servidor de nombre raíz es uno de los trece grupos de servidores
administrados por la IANA para dirigir el tráfico a todos los dominios
de primer nivel, como el núcleo del sistema DNS.

\subsection{Servidor DNS}\label{servidor-dns}

A servidor DNS, o servidor de nombres, es un servidor que proporciona la
función de consulta del sistema de nombres de dominio. Esto se hace ya
sea mediante el acceso a un registro existente en caché de la dirección
IP de un dominio específico, o mediante el envío de una solicitud de
información a otro servidor de nombres.

\subsection{Servidor proxy}\label{servidor-proxy}

Un servidor proxy es un servidor, sistema de computadora o un programa
de aplicación que funciona como pasarela entre un cliente y un servidor
web. Un cliente se conecta al servidor proxy y envía una petición a una
página web desde un servidor diferente. Luego el servidor proxy accede
al recurso conectándose al servidor especificado, y devuelve la
información al sitio solicitante. Los servidores proxy pueden servir
para diferentes propósitos, incluyendo el acceso a páginas web
prohibidas o para ayudar a los usuarios a enrutarse sin obstáculos.

\subsection{Shell (terminal, consola)}\label{shell-terminal-consola}

Un shell de UNIX es una interfaz de usuario de línea de comandos
tradicional para sistemas operativos UNIX y GNU/Linux. Los shells más
comunes son sh y bash.

\subsection{Smartphone (teléfono
inteligente)}\label{smartphone-teluxe9fono-inteligente}

Un smartphone es un teléfono móvil que ofrece capacidades de
conectividad y computación más avanzadas que cualquier teléfono móvil
común contemporáneo, tales como acceso web, capacidad de ejecución de
sistemas operativos elaborados y de aplicaciones integradas.

\subsection{SOCKS}\label{socks}

Un proxy socks es una clase especial de servidor proxy. En el modelo OSI
opera entre las capas de aplicación y de transporte. El puerto estándar
para un proxy SOCKS es 1080, pero puede correr en otros. Algunos
programas soportan una conexión a través de un proxy SOCKS. Otra opción
es instalar un cliente como FreeCap, ProxyCap o SocksCap los cuales
pueden forzar a los programas a correr a través de un proxy Socks usando
reenvío por puerto dinámico. También es posible utilizar herramientas
SSH tales como OpenSSH como un servidor proxy SOCKS.

\subsection{Software de cadena de
claves}\label{software-de-cadena-de-claves}

vea \emph{administración de contraseñas}

\subsection{Spam}\label{spam}

El spam son los mensajes que saturan a un canal de comunicación
utilizado por la gente, sobre todo con publicidad comercial, enviados a
un gran número de individuos o a grupos de discusión. La mayoría del
spam anuncia productos o servicios que son ilegales en una o más formas,
casi siempre incluyendo el fraude. El filtrado de contenidos de correos
electrónicos para bloquear el spam, con el permiso del destinatario, es
una práctica universalmente extendida.

\subsection{SSH (shell seguro)}\label{ssh-shell-seguro}

El SSH o shell seguro es un protocolo de red que permite las
comunicaciones cifradas entre computadoras. Se inventó para suceder al
protocolo sin cifrado Telnet, usado para acceder a un shell en un
servidor remoto.

El puerto estándar SSH es el puerto 22. Puede ser usado para eludir la
censura en Internet con el reenvío de puertos o como túnel de otros
programas tales como VNC.

\subsection{\texorpdfstring{SSL (Secure Sockets Layer, \emph{capa de
conexión
segura})}{SSL (Secure Sockets Layer, capa de conexión segura)}}\label{ssl-secure-sockets-layer-capa-de-conexiuxf3n-segura}

SSL (o Secure Sockets Layer), es un estándar de cifrado usado para
realizar transacciones seguras en Internet. Es la base sobre la cual se
creó el TLS (Transport Layer Security, \emph{capa de transporte
segura}). Puede averiguar fácilmente si está usando SSL observando la
URL en su navegador web (por ejemplo Firefox o Internet Explorer): si
comienza con https en lugar de http, su conexión está cifrada.

\subsection{Subdominio}\label{subdominio}

Un subdominio es una parte de un dominio mayor. Por ejemplo,
``wikipedia.org'' es el dominio de Wikipedia, ``es.wikipedia.org'' es el
subdominio de la versión en español de Wikipedia.

\section{Texto plano}\label{texto-plano}

El texto plano es un texto sin formato que consiste en una secuencia de
códigos de caracteres, como en ASCII o en Unicode.

\subsection{Texto sin formato}\label{texto-sin-formato}

El texto sin formato es texto sin cifrar, o texto descifrado.

vea también \emph{cifrado, TLS/SSL, SSH}.

\subsection{TLS (Seguridad en capa de
transporte)}\label{tls-seguridad-en-capa-de-transporte}

TLS es un estándar de cifrado basado en SSL, usado para realizar
transacciones seguras en Internet.

\subsection{TCP/IP (Protocolo de control de transmisión sobre protocolo
de
Internet)}\label{tcpip-protocolo-de-control-de-transmisiuxf3n-sobre-protocolo-de-internet}

TCP e IP son los protocolos fundamentales de Internet, ya que manejan la
transmisión de paquetes y su ruteo. Existen algunas pocos protocolos
alternativos para ser usados en este nivel de estructura de Internet,
por ejemplo UDP.

\subsection{Túnel}\label{tuxfanel}

Un túnel es una ruta alternativa desde una computadora a otra,
generalmente incluye un protocolo que especifica el cifrado de los
mensajes.

\subsection{Túnel DNS}\label{tuxfanel-dns}

Un túnel DNS es una forma de túnel a través de servidores de nombres
DNS.

Debido a que ``abusa'' del sistema de DNS para un propósito deseado,
sólo se permite una conexión muy lenta de aproximadamente 3 kbs/s que es
incluso menor que la velocidad de un módem analógico. Eso no es
suficiente para YouTube o para compartir archivos, pero debería ser
suficiente para la mensajería instantánea como ICQ o MSN Messenger y
también para el texto sin formato del correo electrónico.

En la conexión que desea utilizar un túnel de DNS, sólo tiene el puerto
53 disponible, pero aún funciona en muchos proveedores comerciales de
Wi-fi sin necesidad de pagar.

El problema principal es que no hay servidores de nombres públicos
modificados que se pueden utilizar. Usted tiene que configurar su
cuenta. Usted necesita un servidor con una conexión permanente a
Internet con Linux. Allí puede instalar el software libre ozymandns y en
combinación con SSH y un proxy como Squid puede utilizar el túnel. Más
información sobre esto en http://www.dnstunnel.de.

\section{UDP (Paquete de datagramas de
usuario)}\label{udp-paquete-de-datagramas-de-usuario}

UDP es un protocolo alternativo usado con IP. Se puede acceder a la
mayoría de los servicios de Internet usando TCP o UDP, pero existen
algunos servicios que están definidos para usar exclusivamente alguno de
los dos. Se usa habitualmente UDP en aplicaciones multimedia en tiempo
real tales como llamadas telefónicas en Internet (VoIP).

\subsection{URL (localizador uniforme de
recursos)}\label{url-localizador-uniforme-de-recursos}

La URL es la dirección del sitio web. Por ejemplo, la URL para la
sección de noticias internacionales del periódico New York Times es
http://www.nytimes.com/pages/world/index.html. Muchos sistemas de
censura pueden bloquear una URL simple. Algunas formas sencillas de
eludir el bloqueo es oscureciendo a la URL. Una manera de hacerlo es
agregando un punto al final del nombre del sitio, entonces la URL
http://en.cship.org/wiki/URL se convierte en
http://en.cship.org./wiki/URL. ; si tiene suerte con este truco podrá
acceder a sitios bloqueados.

\subsection{Usenet}\label{usenet}

Usenet es un sistema de foros de discusión de más de 20 años de
antiguedad al que se accede mediante el protocolo NNTP. Los mensajes no
se almacenan en un servidor pero se encuentran en muchos servidores que
distribuyen su contenido constantemente. Debido a esto es imposible
censurar Usenet como un todo, no obstante el acceso a Usenet puede y se
bloquea a menudo, y cualquier servidor en particular es probable que
lleve sólo un subconjunto de grupos de noticias de Usenet localmente
aceptables. Existen archivos de Google con toda la historia disponible
de mensajes de Usenet para su búsqueda.

\section{VoIP (Protocolo de voz sobre
Internet)}\label{voip-protocolo-de-voz-sobre-internet}

VoIP se refiere a uno de varios protocolos para comunicación entre dos
voces en tiempo real en Internet, que es notoriamente más barata que la
llamada entre redes de voz de compañías telefónicas estándares. Una
ventaja es que no pueden ser objeto de escuchas telefónicas como las
practicadas en la telefonía tradicional, pero pueden ser monitoreadas
usando tecnología digital. Muchas compañías producen software y
equipamiento para escuchar llamadas VoIP; las tecnologías de VoIP
cifrado y seguro recién se están desarrollando.

\subsection{VPN (red privada virtual)}\label{vpn-red-privada-virtual}

Una VPN es una red de comunicación privada usada por muchas empresas y
organizaciones para conectarse en forma segura sobre una red pública.
Generalmente está cifrada y nadie, excepto los extremos de la
comunicación pueden ver el tráfico de datos. Existen varios estándares
tales como IPSec, SSL, TLS. El uso de un proveedor de VPN es un método
muy rápido, seguro y conveniente para eludir la censura en Internet con
bajo riesgo pero generalmente tiene un costo mensual. Sin embargo, tenga
en cuenta que el estándar PPTP no es considerado muy seguro, por lo cual
se desaconseja su uso.

\section{Webmail}\label{webmail}

Webmail es un servicio a través de un sitio web. El servicio envía y
recibe los mensajes de correo para los usuarios de la manera habitual,
pero suministra una interfaz web para leer y administrar mensajes, como
una alternativa al uso de un cliente de correo tal como Outlook Express
o Thunderbird en la computadora del usuario. Por ejemplo, un popular
servicio de webmail libre es https://mail.google.com/

\subsection{WHOIS}\label{whois}

WHOIS (who is, \emph{¿quién es?}) es la función de Internet que permite
realizar consultas a bases de datos WHOIS remotas para obtener
información de registro de dominios. Mediante la realización de una
simple búsqueda WHOIS puede descubrir cuándo y quién ha registrado un
dominio, información de contacto y más.

Una búsqueda WHOIS también puede revelar el nombre o la red mapeada a
una dirección IP numérica.

\subsection{World Wide Web (WWW)}\label{world-wide-web-www}

La World Wide Web es la red de dominios y páginas de contenido con
hipervínculos accesibles usando el protocolo de transferencia de
hipertexto y sus numerosas extensiones. La World Wide Web es la parte
más famosa de Internet.
